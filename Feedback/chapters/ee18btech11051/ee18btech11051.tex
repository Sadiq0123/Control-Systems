\begin{enumerate}[label=\arabic*.,ref=\theenumi]
%\begin{enumerate}[label=\thesection.\arabic*.,ref=\thesection.\theenumi]
\numberwithin{equation}{enumi}

\item

For the circuit in Fig. \ref{fig:ee18btech11051_fig1}, break the loop at node X and find the loop gain (working backward for simplicity to find $V_{X}$ in terms of $V_{O}$). For R = 10 k$\ohm$, find C and $R_{f}$ to obtain sinusoidal oscillations at 10 kHz.

\begin{figure}[!ht]
	\begin{center}
		\resizebox{\columnwidth}{!}{\input{./figs/ee18btech11051/ee18btech11051_fig1.tex}}
	\end{center}
\caption{}
\label{fig:ee18btech11051_fig1}
\end{figure}

\item
\solution
We first calculate the relation between $I_{4}$ and $V_{X}$ in fig \ref{fig:ee18btech11051_fig2} by using the relation between the currents and the fact that the inverting terminal of the Op-Amp is virtually grounded as follows:
\begin{figure}[!ht]
	\begin{center}
		\resizebox{\columnwidth}{!}{\input{./figs/ee18btech11051/ee18btech11051_fig2.tex}}
	\end{center}
\caption{}
\label{fig:ee18btech11051_fig2}
\end{figure}

Here, $V_{4}$ has zero voltage. Applying KVL between $V_{3}$ and $V_{4}$, we get
\begin{align}
    V_{3} = I_{4}R \label{eq:ee18btech11051_0}
\end{align}

Applying KCL and KVL between $V_{2}$ and $V_{3}$, and substituting eq.\ref{eq:ee18btech11051_0} gives
\begin{align}
    I_{3} = I_{4} + V_{3}sC \implies I_{3} = I_{4}(1+sRC) \label{eq:ee18btech11051_1}\\
    V_{2} = V_{3} + I_{3}R
    \implies V_{2} = I_{4}R(2+sRC) \label{eq:ee18btech11051_2}
\end{align}

Using eq.\ref{eq:ee18btech11051_1} and eq.\ref{eq:ee18btech11051_2} in KCL at node $V_{2}$ 
\begin{align}
    I_{2} = I_{3} + V_{2}sC\\
    I_{2} = I_{4}((1+sRC) + sRC(2+sRC))\\
    \implies = I_{2} = I_{4}(1+3sRC+(sRC)^{2}) \label{eq:ee18btech11051_3}
\end{align}

Applying KVL between nodes $V_{1}$ and $V_{2}$, and substituting eq.\ref{eq:ee18btech11051_2} and eq.\ref{eq:ee18btech11051_3} gives
\begin{align}
    V_{1} = V_{2}+ I_{2}R\\
    \implies V_{1} = I_{4}R(3+4sRC+(sRC)^{2})\label{eq:ee18btech11051_4}
\end{align}

Substituting eq.\ref{eq:ee18btech11051_3} and eq.\ref{eq:ee18btech11051_4} in KCL at node $V_{1}$ gives
\begin{align}
    I_{1} = I_{2} + V_{1}sC\\
    \implies I_{1} = I_{4}(1+6sRC+5(sRC)^2+(sRC)^3)\label{eq:ee18btech11051_5}
\end{align}

KVL between $V_{X}$ and $V_{1}$, and using eq.\ref{eq:ee18btech11051_4} and eq.\ref{eq:ee18btech11051_5} gives 
\begin{align}
    V_{X} = V_{1} + I_{1}R\\
    \implies V_{X} = I_{4}R(4+10sRC+6(sRC)^2+(sRC)^3)
    \label{eq:ee18btech11051_6}
\end{align}

Substituting $ s = j\omega$ in the eq.\ref{eq:ee18btech11051_6}:
\begin{align}
    V_{X} = I_{4}R((4-6(\omega RC)^{2}) + j(10\omega RC - (\omega RC)^{3})) \label{eq:ee18btech11051_7}
\end{align}
The relation between $I_{4}$ and $V_{O}$ is:
\begin{align}
    V_{O} = -I_{4} R_{f}
\end{align}
So, the transfer function GH from eq.\ref{eq:ee18btech11051_7} is
\begin{align}
     GH = \frac{V_{O}}{V_{X}} = -\frac{R_{f}}{R((4-6(\omega RC)^{2}) + j(10\omega RC - (\omega RC)^{3}))} \label{eq:ee18btech11051_8}
\end{align}
To make the system oscillate at a frequency $f_{O}$,
\begin{align}
    |GH|=1\\ \angle GH = 0^{\degree} \\
    \implies (\omega RC)^{2} = 10 \label{eq:ee18btech11051_9}\\
    \implies C = \frac{\sqrt{10}}{2\pi f_{O}R} = 5.03nF
\end{align}
Putting GH = 1 in eq.\ref{eq:ee18btech11051_8} gives:
\begin{align}
    -\frac{R_{f}}{R(4-6(10))} = 1 \implies R_{f} = 56R = 560k\ohm
\end{align}
Therefore, the values of C and $R_{f}$ for the desired oscillator are 5.03nF and 560k$\ohm$.
\begin{table}[!ht]
    \centering
    \input{./tables/ee18btech11051/ee18btech11051_table1.tex}
    \caption{Final Values}
    \label{table:ee18btech11051_table_1}
\end{table}

\item
Verification:\\
The value of $R_{f}$ is kept slightly above the calculated value for the system to start oscillating




\end{enumerate}
